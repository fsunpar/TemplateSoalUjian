%template by: lnv

\documentclass[11pt,a4paper]{article}
\usepackage[utf8]{inputenc}
\usepackage{geometry}
\usepackage{graphicx}
\usepackage{charter}
\usepackage{xspace}
\usepackage{lipsum}
\usepackage{xstring}
\usepackage{enumitem}
\usepackage{booktabs}
\usepackage{wrapfig}

\geometry{left=1.25cm,right=1.25cm,bottom=1.5cm,top=0.75cm}
\newcommand{\bfu}[1]{{\bf \MakeUppercase{#1}}}
\newcommand{\nilai}[1]{\IfStrEq{#1}{}{}{{\bf [#1]}}}
\newcommand{\soal}[2]{\nilai{#1} #2}

%% =================== ISI INFORMASI DI SINI ==============================
\newcommand{\ujian}{UTS}    %UTS atau UAS
\newcommand{\tahun}{1598/1662}
\newcommand{\semester}{GANJIL} %GANJIL/GENAP
\newcommand{\kode}{Dodo101}
\newcommand{\namaMK}{Introduction to Dodo}
\newcommand{\prodi}{Madagaskar}       %Matematika/Fisika/Informatika
\newcommand{\sifat}{buka}   %buka/tutup (case sensitive)
\newcommand{\hari}{KAMIS}
\newcommand{\tanggal}{1 Januari 1970}
\newcommand{\waktu}{560640}      %dalam menit
\newcommand{\dosens}{Dodo} %nama dosen(-dosen)
\newcommand{\petunjuk}{}    %kosongkan jika tidak digunakan (SPASI == tidak kosong!!). Termasuk apakah boleh pakai kalkulattor atau tidak.

%Dosen Ajar
%Petunjuk Kerja

%% ========================= DOKUMEN UJIAN ====================================
\begin{document}

%% ========================= HEADER ===========================================
\begin{minipage}{0.125\textwidth}\includegraphics[scale=0.3]{unparpng}\end{minipage}\begin{minipage}{0.7\textwidth}%\begin{center}
{\bf \large FAKULTAS SAINS\\
%UNIVERSITAS KATOLIK PARAHYANGAN\\
UJIAN \IfStrEq{\ujian}{UAS}{TENGAH}{AKHIR} SEMESTER \MakeUppercase{\semester} \\TAHUN AKADEMIK \tahun}
%\end{center}
\end{minipage}

\renewcommand{\arraystretch}{1.25}
\begin{center}\begin{tabular} {l l p{11.8cm}}\midrule \toprule \bfu{Kode \& Nama Mata Kuliah}	&:& \kode\ -- \namaMK \\\bfu{Program Studi}&:& \prodi \\\bfu{Sifat Ujian} &:& \IfStrEq{\sifat}{buka}{\bfu{buka}}{\bfu{tutup}} \bfu{BUKU}\\\bfu{Hari/Tanggal} &:& \hari/\tanggal \\\bfu{Waktu Pengerjaan} &:& \waktu\ \bfu{MENIT}\\\bfu{Dosen} &:& \dosens\\ \bottomrule \end{tabular}\end{center}

\IfStrEq{\petunjuk}{}{}{\begin{minipage}{0.1\textwidth} {\bf Petunjuk:}\end{minipage} \begin{minipage}{0.85\textwidth} \petunjuk \end{minipage}}

%%==================== TULIS SOAL DI SINI ==================================
%tambahkan sendiri jika perlu
%format: \soal{nilai}{soal} --> kosongkan nilai jika tidak diperlukan, SPASI==tidak kosong!!!

\begin{enumerate}
    \item
    \soal{10}{%SOAL 1
        {\scshape Ini soal nomor satu}. \lipsum[1]
    }
    \item
    \soal{10}{%SOAL 2
        Jawablah pertanyaan-pertanyaan berikut ini
        \begin{enumerate}[noitemsep,nolistsep]
            \item Pertanyaan a
            \item Pertanyaan b
        \end{enumerate}
    }
    \item
    \soal{10}{%SOAL 3
        {\scshape Ini soal nomor empat}.
    }
    \item
    \soal{10}{%SOAL 4
        {\scshape Ini soal nomor iv}.
    }    
    \item
    \soal{}{%SOAL 5
        {\scshape Ini soal nomor 5}.
    }    
\end{enumerate}

\begin{center}
    { --- --- --- *** --- --- ---}
\end{center}

\end{document}
